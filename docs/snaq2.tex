\documentclass{article}

\title{SNaQ 2.0}
\author{Nathan Kolbow, Sungsik Kong \& Claudia Solís-Lemus}
\date{\today}

\usepackage{xcolor}
\newcommand{\kevin}{\color{blue}}

\begin{document}
\maketitle

The goal of this study is to introduce a new software, SNaQ 2.0, which is an updated version of original SNaQ with an improvement in computational cost. In this study, we want to emphasize that SNaQ 2.0 is faster than the original SNaQ with their accuracy being equal.

\section{Pipeline}

{\kevin We must start thinking about how to use HTC for this project.}

\subsection{Species network generation}
We decided to create 9 different network topologies that varies in size and complexity, controlled by the number of tips $n$ and reticulations $r$ in the network. We set $n \in \{10, 25, 50\}$ and $r \in \{1, 3, 5\}$. We generate the species networks semi-manually. Currentl plan is to produce tree topology under some model (i.e., Yule process) and sequentially as reticulations onto the topology. But there might be a better way. 

\subsection{Gene tree and sequence generation}
For each species network, we use \texttt{PhyloCoalSimulations} to generate $g$ gene trees. We set $g \in \{50, 100, 500, 1000, 3000\}$. Then, we use $seq-gen$ to generate sequence alignment for each of the gene trees {\kevin We have to decide sequence length and some other parameters like which evolutionary model we will use.}

\subsection{Gene tree estimation}
For each simulated sequence alignment, we estimate a gene tree using \texttt{iq-tree 2}. We then compute gene tree estimation error (GTEE) as well.

\subsection{Network estimation}
Using the set of gene trees, we estimate species network using SNaQ 1.0 and SNaQ 2.0. We assume the true number of reticulations in the final network is known.

\subsection{Analysis}
To analysize the accuracy, we compare each of the estimated species network with the true topology. The network dissimilarity is measured using hardwired cluster distance. 

To analyze the computational cost, we can use the time taken reported in the output of SNaQ. Alternatively, may be we can use CPU time reported by the cluster.

We use \texttt{ggplot2} to plot the output.

\subsection{Empirical datasets}
First option is to replicate the swordflish network reported in the SNaQ 1.0 paper.

We might want to use another empirical dataset. Some preferences are the dataset of the biological system where hybridization is well studied, contains estimated set of gene trees, and quite reliable.

\subsection{Submission}
Currently, software note on Bioinformatics or Methods in Ecology and Evolution seem to be feasible options. We shoot for late December to do this.


\end{document}